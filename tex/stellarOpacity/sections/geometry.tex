
  \section{General}
  To my knowledge, Cartesian coordinates are not fundamental in any objective
    way, they were just the first we formalized and so we consider them easy
    and pretend to have a robust understanding of their properties.

  For example, Cartesian coordinate representations of any vector is unique or
    non-degenerate.
  This is nice, but only true for some coordinate systems!
  This is also very important, we don't want to be able to represent systems in
    non-unique ways, any objective measure would be rendered useless.
  Equivalently, perhaps
    arithmetic based on degenerate coordinate systems might not have led us to
    such a good language to interpret the natural world.
  So how do we identify these non-degenerate coordiante systems?

  Well, we say a coordinate system is non-degenerate (like Cartesian) if
    we can demonstrate that the other coordinate system is isomorphic to
    the Cartesian Coordinate System.
  By `isomorphic`, I mean that there is a mapping from one space to another
      that can be exactly reversed by an inverse mapping.
  In the language of coordinate systems, we use metric tensors to figure this
    out.


  \subsection{Metric Tensor}\label{s:metricTensor}
  Super broadly, a metric tensor is a function that appears in differential
    geometry.
  The metric tensor gives definition to generalized mathematical spaces
    (called manifolds) by relating real-number scalars to distances and
    angles (in the case of positive-definite metric tensors).

  One way to think of a metric tensor is as a rank-2 matrix, $g$.
  For a Cartesian coordinate system, $g_{ij} = \delta_{ij}$ where $\delta$ is
    the Kroenecker Delta function.
  Without proof (again, sorry), I claim that if you can characterize a
    coordinate system $A$ with metric tensor $g^A$, you can show that it is
    isomorphic to the Cartesian Coordinate System by demonstrating that the
    metric tensor is diagonalizable.
  So, if a metric tensor is diagonalizable, there is an isomorphism between
    that coordinate system and the Cartesian Coordinate system\footnote{ If the
    metric is only locally diagonalizable, the manifold is only locally
    isomorphic to the Cartesian Metric.}.
  If a coordinate system is isomorphic to Cartesian, then it must also be
    non-degenerate.

  I'd like to drive home the importance of this in terms of mathematical
    representations of physical systems. Again, isomorphisms are great!
  They mean that we can move back and forth between representations
    without losing information.

  For example, you are probably aware that the 2-D map of the surface of the
    (spherical) Earth *cannot* be perfectly mapped to a 2-D Cartesian
    representation.
  However! We can make arbitrarily accurate maps of small patches of the Earth.
  This is because spherical coordinates define a representation
    that is only locally diagonalizable in terms of its metric tensor.

    \subsection{Scale Factors}\label{s:scaleFactors}
    WHY IN THE 3-D CARTESIAN WORLD AM I BOTHERING WITH THIS?

    Well, the various indeces of a metric are related to what are known as the
      scale factors. Scale factors (in the context of metric tensors) tell you
      how space transforms with respect to the coordinate parameters.
    This idea is super important in general relativity when we introduce the
      additional coordinate paramter $t$ and see that it depends on (e.g.) the
      rate of change of the other coordinate parameters.

    Happily, I am not going to forray into GR, but I will show you where else
      scale factors play an important roles that has almost certainly between
      glossed-over in your introductory physics and vector calculus courses.

    For a diagonal (or even locally diagonal) metric tensor,
      $g_{ij} = g_{ii}\delta_{ij}$ with the coordinate parameterization

      \begin{align*}
        x_1 &= f_1(q_1, q_2, \cdots, q_n)\\
        x_2 &= f_2(q_1, q_2, \cdots, q_n)\\
        &\cdots \\
        x_n &= f_n(q_1, q_2, \cdots, q_n)
      \end{align*}

    The scale factor($h_i$) is simply the square root of the diagonal metric element!

    That is,

    \begin{equation}
      h_i = \sqrt{g_{ii}} = \sqrt{\sum_{k=1}^n\Bigl(\frac{\partial x_k}{\partial q_i}\Bigr)^2}\label{eqn:scaleEqn}
    \end{equation}

    Here's the fun part (for me ...)

    These scale factors allow us to transform any coordinate system differential
      elements. Of particular use for this study is the line element,
      the area element, and the volume element.

    Taking each in turn ...
    \paragraph{Line Element}
      The line element is given in terms of scale factors and the above
        coordinate parameterization as

    \begin{equation}
      \dd{\vec{l}} = h_1\dd{q_1}\hat{q_1} + h_2\dd{q_2}\hat{q_2} + \cdots + h_n\dd{q_n}\hat{q_n}\label{eqn:lineElem}
    \end{equation}

    \paragraph{Area Element}
      The area element is given in terms of scale factors and the above
        coordinate parameterization as

    \begin{equation}
      \dd{^2\vec{s}_{ab}} = h_ah_b\dd{q_a}\dd{q_b}\cdot (\hat{q_a}\times \hat{q_b})\label{eqn:areaElem}
    \end{equation}

      Where the cross product appears in order to orient the area element with
        the generalized cross-product.

    \paragraph{Volume Element}
      The volume element is given in terms of scale factors and the above
        coordinate parameterization as

    \begin{equation}
      \dd{^3V} = h_1h_2h_3\dd{q_1}\dd{q_2}\dd{q_3}\label{eqn:volumeElem}
    \end{equation}

      \section{Cartesian}\label{s:cartScaleFactors}
        Let's quickly make use of these to define the differential elements of
        our favorite simple coordinate system.

        The Cartesian Metric Tensor is trivially diagonal. Therefore,
          $g_{ij} = g_{ii}\delta_{ij}$ with the coordinate parameterization

          \begin{align*}
            x_1 &= x\\
            x_2 &= y\\
            x_3 &= z
          \end{align*}

          Recall equation \ref{eqn:scaleEqn}
          $$
          h_i = \sqrt{\sum_{k=1}^n\Bigl(\frac{\partial x_k}{\partial q_i}\Bigr)^2}
          $$

          \begin{align*}
            h_x &= \sqrt{\sum_{k=1}^3\Bigl(\frac{\partial x_k}{\partial q_i}\Bigr)^2}\\
            h_x &= \sqrt{\Bigl(\frac{\partial x}{\partial x}\Bigr)^2 +
              \Bigl(\frac{\partial y}{\partial x}\Bigr)^2 +
              \Bigl(\frac{\partial z}{\partial x}\Bigr)^2}\\
            h_x &= \sqrt{\Bigl(1\Bigr)^2 +
              \Bigl(0\Bigr)^2 +
              \Bigl(0\Bigr)^2}\\
            h_x &= 1
          \end{align*}

          The rest are easily seen to be $1$ as well!\\
          \paragraph{Cartesian Scale Factors}
          $h_x = 1; h_y = 1; h_z = 1$

          Now, recalling equations \ref{eqn:lineElem}, \ref{eqn:areaElem}, \ref{eqn:volumeElem} ...
          \begin{equation}
            \dd{\vec{l}} = \dd{x}\hat{x} + \dd{y}\hat{y} + \dd{z}\hat{z}\label{eqn:cartLineElem}
          \end{equation}

          \begin{equation*}
            \dd{^2\vec{s}_{xy}} = \dd{x}\dd{y} \cdot (\hat{x}\times \hat{y})
          \end{equation*}
          \begin{equation}
            \dd{^2\vec{s}_{xy}} = \dd{x}\dd{y} \cdot \hat{z}\label{eqn:cartAreaElem}
          \end{equation}
          Note: Other combinations of the coordinates would generate other area elements!
          This will be sometimes noted as an oriented area element or a
            vector area element to distinguish it from uses of its magnitude.

          \begin{equation}
            \dd{^3V} = \dd{x}\dd{y}\dd{z}\label{eqn:cartVolumeElem}
          \end{equation}

      Easy!

      \section{Spherical}\label{s:sphericalScaleFactors}
      The Spherical Metric Tensor is only \emph{locally} diagonal. In spite of
        this, we can still make use of $g_{ij} = g_{ii}\delta_{ij}$
        \emph{locally} with regard to differential paramters. And define the
        parameterization of the Cartesian Coordinates in terms of the Spherical
        parameters: the radial distance, the polar angular distance, and the
        azimuthal angular distance ($r, \theta, \phi$).

        Figures \ref{fig:cart2Sph} and \ref{fig:spherical} give a schematic
          for how Cartesian and Spherical Coordinate Systems are related.

        The Cartesian Coordinate $x$ can be seen to be the radial distance
          from the origin scaled and projected onto the $x-y$ plane. That is:
          $x = r\sin{\theta}\cos{\phi}$.

        Similarly, $y = r\sin{\theta}\sin{\phi}$ and $z = r\cos{\theta}$.

        \begin{figure}[h]
          \centering
          \includegraphics[width=300px]{cartCoords}
          \caption{\label{fig:cart2Sph} Along with }

          \includegraphics[width=300px]{sphericalCoords}
          \caption{\label{fig:spherical} Along with }
        \end{figure}

        In this parameterization below, note that
          $(x_1, x_2, x_3) \rightarrow (x, y, z)$.
        \begin{align*}
          x_1 &= r\sin{\theta}\cos{\phi}\\
          x_2 &= r\sin{\theta}\sin{\phi}\\
          x_3 &= r\cos{\theta}
        \end{align*}

        Recall equation \ref{eqn:scaleEqn}
        $$
        h_i = \sqrt{\sum_{k=1}^n\Bigl(\frac{\partial x_k}{\partial q_i}\Bigr)^2}
        $$
        With the

        \begin{align*}
          h_r &= \sqrt{\sum_{k=1}^3\Bigl(\frac{\partial x_k}{\partial r}\Bigr)^2}\\
          h_r &= \sqrt{\Bigl(
            \frac{\partial x_1}{\partial r}\Bigr)^2
            +\Bigl(\frac{\partial x_2}{\partial r}\Bigr)^2
            +\Bigl(\frac{\partial x_3}{\partial r}\Bigr)^2}\\
          h_r &= \sqrt{\Bigl(
            \frac{\partial}{\partial r}r\sin{\theta}\cos{\phi}\Bigr)^2
            +\Bigl(\frac{\partial}{\partial r}r\sin{\theta}\sin{\phi}\Bigr)^2
            +\Bigl(\frac{\partial}{\partial r}r\cos{\theta}\Bigr)^2}\\
          h_r &= \sqrt{
             \sin{^2\theta}\cos{^2\phi}\Bigl(\frac{\partial}{\partial r}r\Bigr)^2
            +\sin{^2\theta}\sin{^2\phi}\Bigl(\frac{\partial}{\partial r}r\Bigr)^2
            +\cos{^2\theta}\Bigl(\frac{\partial}{\partial r}r\Bigr)^2}\\
          h_r &= \sqrt{
             \sin{^2\theta}\cos{^2\phi}\Bigl(1\Bigr)
            +\sin{^2\theta}\sin{^2\phi}\Bigl(1\Bigr)
            +\cos{^2\theta}\Bigl(1\Bigr)}\\
          h_r &= \sqrt{\sin{^2\theta}(\underbrace{\cos{^2\phi} +\sin{^2\phi}}_{1; Pythagorean Identity}) +\cos{^2\theta}}\\
          h_r &= \sqrt{\sin{^2\theta} +\cos{^2\theta}} = \sqrt{1^2} = 1
        \end{align*}

        \begin{align*}
          h_\theta &= \sqrt{\sum_{k=1}^3\Bigl(\frac{\partial x_k}{\partial r}\Bigr)^2}\\
          h_\theta &= \sqrt{\Bigl(
            \frac{\partial x_1}{\partial \theta}\Bigr)^2
            +\Bigl(\frac{\partial x_2}{\partial \theta}\Bigr)^2
            +\Bigl(\frac{\partial x_3}{\partial \theta}\Bigr)^2}\\
          h_\theta &= \sqrt{\Bigl(
            \frac{\partial}{\partial \theta}r\sin{\theta}\cos{\phi}\Bigr)^2
            +\Bigl(\frac{\partial}{\partial \theta}r\sin{\theta}\sin{\phi}\Bigr)^2
            +\Bigl(\frac{\partial}{\partial \theta}r\cos{\theta}\Bigr)^2}\\
          h_\theta &= \sqrt{
            r^2\cos{^2\phi}\Bigl(\frac{\partial}{\partial \theta}\sin{\theta}\Bigr)^2
            +r^2\sin{^2\phi}\Bigl(\frac{\partial}{\partial \theta}\sin{\theta}\Bigr)^2
            +r^2\Bigl(\frac{\partial}{\partial \theta}\cos{\theta}\Bigr)^2}\\
          h_\theta &= r\sqrt{\Bigl(
            \cos{^2\phi}\Bigl(\cos{\theta}\Bigr)^2
            +\sin{^2\phi}\Bigl(\cos{\theta}\Bigr)^2
            +\Bigl(-\sin{\theta}\Bigr)^2}\\
          h_\theta &= r\sqrt{
            \cos{^2\phi}\cos{^2\theta}
            +\sin{^2\phi}\cos{^2\theta}
            +\sin{^2\theta}}\\
          h_\theta &= r\sqrt{
            \cos{^2\theta}(\underbrace{\cos{^2\phi} +\sin{^2\phi}}_{1; Pythagorean Identity})
            +\sin{^2\theta}}\\
          h_\theta &= r\sqrt{
            \cos{^2\theta} +\sin{^2\theta}} = r \sqrt{1^2}
        \end{align*}

        Do the $\phi$ one! You should get $r\sin{\theta}$.

        \paragraph{Spherical Scale Factors}
        $h_r = 1; h_\theta = r; h_\phi = r\sin{\theta}$

        Now, recalling equations \ref{eqn:lineElem}, \ref{eqn:areaElem}, \ref{eqn:volumeElem} ...
        \begin{equation*}
          \dd{\vec{l}} = h_r\dd{r}\hat{r} + h_\theta\dd{\theta}\hat{\theta} + h_\phi\dd{\phi}\hat{\phi}
        \end{equation*}
        \begin{equation}
          \dd{\vec{l}} = \dd{r}\hat{r} + r\dd{\theta}\hat{\theta} + r\sin{\theta}\dd{\phi}\hat{\phi}\label{eqn:spherelineElem}
        \end{equation}

        \begin{equation*}
          \dd{^2\vec{s}_{\theta\phi}} = h_\theta h_\phi \dd{\theta}\dd{\phi}\cdot (\hat{\theta}\times \hat{\phi})
        \end{equation*}
        \begin{equation*}
          \dd{^2\vec{s}_{\theta\phi}} = r \cdot r\sin{\theta} \dd{\theta}\dd{\phi} \cdot \hat{r}
        \end{equation*}
        \begin{equation}
          \dd{^2\vec{s}_{\theta\phi}} = r^2\sin{\theta} \dd{\theta}\dd{\phi}\hat{r}\label{eqn:sphereAreaElem}
        \end{equation}
        Note: Other combinations of the coordinates would generate other area elements!
        Also note that $\dd{^2\vec{s}_{\theta\phi}}$ is oriented away from the surface of the sphere, that is: in the radial direction.
        This will be sometimes noted as an oriented area element or a vector area element to distinguish it from uses of its magnitude

        \begin{equation*}
          \dd{^3V} = h_r h_\theta h_\phi\dd{r}\dd{\theta}\dd{\phi}
        \end{equation*}
        \begin{equation}
          \dd{^3V} = r^2\sin{\theta}\dd{r}\dd{\theta}\dd{\phi}\label{eqn:sphereVolumeElem}
        \end{equation}

        These definitions will be extemely useful moving forward!

        \subsubsection{Solid Angle}\label{s:solidAngle}
          To start the discussion with solid angles, let us actually begin with a surface integral over the entire area of a sphere. We will use the definition of the spherical area element given in
            \ref{eqn:sphereAreaElem}.

          Recall that $\dd{^2\vec{s}_{\theta\phi}}$ is a vector quantity.
          We will just take the magnitude of this oriented area element.

          $$
            \int_{sphere}|\dd{^2\vec{s}_{\theta\phi}}|
          $$

          In order to cover the entire sphere, $\theta$ must vary from $0$ (at the top of the sphere in \ref{fig:spherical}) to $\pi$ and$\phi$ must vary from $0$ to $2\pi$ along the great circle containing the $x-y$ plane.

          $$
            \int_{\phi=0}^{2\pi}\int_{\theta=0}^{\pi}r^2\sin{\theta} \dd{\theta}\dd{\phi}
          $$

          What we're left with is actually fairly easy to solve!
          First, we can pull $r^2$ out of the integral, since it does not vary with $\theta$ or $\phi$.
          Additionally, we can separate the two integrals fairly cleanly!

            $$
              r^2\int_{\phi=0}^{2\pi}\dd{\phi} \cdot \int_{\theta=0}^{\pi}\sin{\theta} \dd{\theta}
            $$

            The $\phi$ integral readily goes to $2\pi$.
            $$
              r^2 \cdot 2\pi \cdot \int_{\theta=0}^{\pi}\sin{\theta} \dd{\theta}
            $$

            The $\theta$ integral is only \emph{slightly} more complicated, but happily yields $2$.
            $$
              r^2 \cdot 2\pi \cdot 2
            $$

            I will repeat that $\theta$ integral with a substitution that will be later employed.
            $\mu = \cos{\theta}$ and $\dd{\mu} = -\sin{\theta}\dd{\theta}$
            $$
              r^2 \cdot 2\pi \cdot \int_{\mu(0)=1}^{\mu(\pi)=-1}-\dd{\mu}
            $$
            Flip the bounds to remove the $-$ sign ...
            $$
              r^2 \cdot 2\pi \cdot \int_{\mu=-1}^{1}\dd{\mu}
            $$
            And it's even easier to see that we end up with $2$.

            Taking the product, we end up with the fact that the surface area of a sphere is $4\pi r^2$. Agreed? GOOD.

            Now, that $r^2$ actually gets us into a lot of trouble. In some weird way, it specifies an otherwise generalizable sphere.

            If we strip the $r^2$ from the $\dd{^2\vec{s}_{\theta\phi}}$, we're actually left with an intriguing quantity that depends only on two angular measures, $\theta$ and $\phi$.

            \begin{equation}
              \frac{|\dd{^2\vec{s}_{\theta\phi}}|}{r^2} = \sin{\theta}\dd{\theta}
              \dd{\phi} = \dd{^2\Omega} \label{eqn:solidAngle}
            \end{equation}

            It is important to note that the \emph{differential} solid angle element implies a direction, namely $\hat{n}$. This can be made explicit with the inclusion of $\hat{n}$ with the notation: $\dd{^2\Omega}\hat{n}$

            By comparison to the surface integrals performed above, the integral of the differential solid angle over the entire sphere ($\theta \in [0, \pi] , \phi \in [0, 2\pi]$) yields $4\pi$.

            \begin{equation}
              \int_{sphere}\dd{^2\Omega} = 4\pi \label{eqn:solidAngleSphere}
            \end{equation}

            In analogy with the radian for measure of a 1-D angle, the quantity we discuss here is built from a 2-D angle. It therefore is owed a new measure: $rad^2 = sr$--the steradian.

            Just as one might claim a circle has $2\pi$ radians, a sphere has $4\pi$ steradians.

            We'll see that defining quantities in terms of the solid angle and the differential element thereof allows for the calculation of important quantities that arise in astrophysics and astrophysical observation.
