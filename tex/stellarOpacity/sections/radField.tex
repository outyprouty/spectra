\section{Specific Intensity}\label{s:specificInt}

  The Specific Intensity is a differential, scalar quantity \footnote{In radiometry, this quantity is called the spectral radiance.}:

  $$
  \frac{\dd I}{\dd \lambda} = I_\lambda(\vec{r}, \hat{n}, \lambda, t) =
    \frac{\delta E}{\dd{\lambda} \dd{t} \dd{^2 S} \cos{\theta}\dd{^2\Omega} }
  $$

  Where $\delta E$ is the amount of energy transported by radiation belonging to the wavelength range $[\lambda, \lambda + \dd{\lambda}]$.
  Equivalently ...
    $$
    \delta E \approx E_\lambda \dd{\lambda} = \dd{E}
    $$

  In the arguments of $I_\lambda$ reside the location vector $\vec{r}$ locating the oriented area element $\dd{^2 S}$, $\hat{n}$ representing the direction or orientation of the differential solid angle element, $\lambda$ the spectral location, and time $t$.

  The specific intensity therefore represents the amount of spectral, radiative energy that passes perpendicularly across an oriented area element ($\dd{^2\vec{S}}$) in unit time ($\dd{t}$) confined to a differential solid angle element ($\dd{^2\Omega}$).

  The units of the specific intensity are $\frac{W}{m^2\AA sr}$.


  Confining the energy to the differential solid angle implicates a propogation direction for this specific intensity, namely $\hat{n}$ as discussed in section \ref{s:solidAngle}.

  The $\cos{\theta}$ term in the denominator owes to the projection of the differential area element onto the direction of energy propogation.
  That is, the general oriented area element can be written as $\dd{^2\vec{S}} = \dd{^2S}\cdot \hat{s}$ with $\hat{s}$ representing the unit normal to the area element.
  We discussed this notation in spherical coordinates (i.e., \ref{eqn:sphereAreaElem}).
  For reasons we will discuss below, it is generally convienient to align the direction of propogation to the polar axis (or $z$-axis).
  Therefore, $\dd{^2\vec{S}} \dd{^2\Omega}$ is equivalent to $\dd{^2S}\dd{^2\Omega}\hat{s}\cdot\hat{n}$ and finally $\dd{^2S}\dd{^2\Omega}\cos{\theta}$.
  With $\theta$ defining the angle between the propogation direction ($\hat{n}$) and the orientation of the oriented area element ($\hat{s}$).

\subsection{Invariance}\label{s:invariantIntensity}
  Importantly, the use of the solid angle in the development of the specific intensity enables the quantity to be independent of the distance between the source and the observer\footnote{Provided there are no sources or sinks of radiant energy along $\hat{n}$}.

  As an illustrative example without illustration, take $\delta E = I_\lambda(\vec{r}, \hat{n}, \lambda, t)\dd{\lambda} \dd{t} \dd{^2 S} \cos{\theta}\dd{^2\Omega}$ as the amount of spectral, radiative energy that passes perpendicularly across an oriented area element $\dd{^2\vec{S}}$ in unit time confined to a solid angle element $\dd{^2\Omega}$.

  Some distance $r$ away from $\dd{^2\vec{S}}$ is another oriented area element $\dd{^2\vec{S'}}$. As seen from $\dd{^2\vec{S'}}$, $\dd{^2\vec{S}}$ subtends a solid angle element $\dd{^2\Omega}$ and from $\dd{^2\vec{S}}$, $\dd{^2\vec{S'}}$ subtends a solid angle element $\dd{^2\Omega'}$.

  That is, $\dd{^2\Omega} = \frac{\dd{^2\vec{S'}}}{r^2}$ and $\dd{^2\Omega'} = \frac{\dd{^2\vec{S}}}{r^2}$, both by our definition of solid angle \ref{eqn:solidAngle}.

  So, by conservation of energy, $\delta E = I_\lambda(\vec{r}, \hat{n}, \lambda, t)\dd{\lambda} \dd{t} \dd{^2 S}\cos{\theta}\dd{^2\Omega} = I'_\lambda(\vec{r}, \hat{n}, \lambda, t)\dd{\lambda} \dd{t} \dd{^2 S'} \cos{\theta'}\dd{^2\Omega'}$.

  And $I_\lambda = I'_\lambda$.

  Therefore, the specific intesity is spatially invariant.

\section{Flux-Vector}\label{s:fluxVector}

  Flux is a vector-valued quantity representing the amount of spectral, radiative energy that passes perpendicularly across an oriented area element in unit time.

  The flux is measured in units of $\frac{W}{m^2 \AA}$.

  \begin{equation}
    \vec{\mathcal{F}}_\lambda(\vec{r}, \lambda, t) = \int_\Omega I_\lambda(\vec{r}, \lambda, t, \hat{n})\dd{^2\Omega}\hat{n} \label{eqn:totalFlux}
  \end{equation}

  In the above, it is the implied propogation direction $\hat{n}$ associated with $\dd{^2\Omega}$ that yields a vector-valued quantity.

  With the $\vec{\mathcal{F}}_\lambda(\vec{r}, \lambda, t)$ representable as a Cartesian vector:

  $$
    \vec{\mathcal{F}}_\lambda = \mathcal{F}_{\lambda,x}\hat{x} + \mathcal{F}_{\lambda,y}\hat{y} + \mathcal{F}_{\lambda,z}\hat{z}
  $$

  Dropping the spectral ($\lambda$) subscript for ease only for the moment...
  $$
    \vec{\mathcal{F}} = \mathcal{F}_x\hat{x} + \mathcal{F}_y\hat{y} + \mathcal{F}_z\hat{z}
  $$

  $$
    \vec{\mathcal{F}} = \int_\Omega I_\lambda\dd{^2\Omega}\hat{n}\cdot\hat{x}
    + \int_\Omega I_\lambda\dd{^2\Omega}\hat{n}\cdot\hat{y}
    + \int_\Omega I_\lambda\dd{^2\Omega}\hat{n}\cdot\hat{z}
  $$

  At this point it is useful to make use of the substitution $\mu = \cos{\theta}$ and $\dd{\mu} = -\sin{\theta}\dd{\theta}$.

  Therefore,
  \begin{align*}
    \hat{n}\cdot\hat{x} &= \sqrt{(1-\mu^2)}\cos{\phi}\\
    \hat{n}\cdot\hat{y} &= \sqrt{(1-\mu^2)}\sin{\phi}\\
    \hat{n}\cdot\hat{z} &= \mu\\
  \end{align*}

  Substitute these relations into the previous flux vector integrals to write ...

  $$
    \vec{\mathcal{F}} =
    \int_\Omega I_\lambda\cos{\phi}\sqrt{(1-\mu^2)}\dd{^2\Omega}
    + \int_\Omega I_\lambda\sin{\phi}\sqrt{(1-\mu^2)}\dd{^2\Omega}
    + \int_\Omega I_\lambda\mu\dd{^2\Omega}
  $$

  Evaluating each component over the entire $4\pi$ steradians of a sphere ...

  \begin{align*}
    \vec{\mathcal{F}}\cdot\hat{x} &= \int_{\phi=0}^{2\pi}\int_{\mu=-1}^{1} I_\lambda(\vec{r}, \lambda, t, \mu, \phi)\cos{\phi}\sqrt{(1-\mu^2)}\dd{\mu}\dd{\phi}\\
    \vec{\mathcal{F}}\cdot\hat{y} &= \int_{\phi=0}^{2\pi}\int_{\mu=-1}^{1} I_\lambda(\vec{r}, \lambda, t, \mu, \phi)\sin{\phi}\sqrt{(1-\mu^2)}\dd{\mu}\dd{\phi}\\
    \vec{\mathcal{F}}\cdot\hat{z} &= \int_{\phi=0}^{2\pi}\int_{\mu=-1}^{1} I_\lambda(\vec{r}, \lambda, t, \mu, \phi)\mu\dd{\mu}\dd{\phi}
  \end{align*}

  This is as far as we can go before giving more definite form to the specific intensity of a particular system.

\section{Radiative Flux}\label{s:flux}

  In the context of small portions of [stellar] atmospheres, we can make a single simpliying assumption.
  Consider a plane-parallel atmosphere perhaps not unlike the volume element shown schematically in \ref{fig:cart2Sph}.
  If the planar atmosphere is homogenous in the $x-y$ plane, only  $\vec{\mathcal{F}}\cdot\hat{z}$ can be non-zero. Since the scalar radiation field ($I_\lambda(\vec{r}, \lambda, t, \mu, \phi)$) would be symmetric with respect to the $z$ axis, there will be ray-by-ray cancellation in the net flux across a surface perpendicular to that axis of symmetry (the $x-y$ plane).
  It is here that we decide to align the propogation direction with this axis of symmetry and slightly modify the specific intensity.

  $$
    I_\lambda(\vec{r}, \lambda, t, \mu, \phi) \rightarrow I_\lambda(\vec{r}, \lambda, t, \mu)
  $$

  In terms of the integrals above, we take advantage of this simplification as follows:

  \begin{align*}
    \vec{\mathcal{F}}\cdot\hat{x} &= \underbrace{\int_{\phi=0}^{2\pi}\cos{\phi}\dd{\phi}}_{0}
    \int_{\mu=-1}^{1} I_\lambda(x, \lambda, t, \mu)\sqrt{(1-\mu^2)}\dd{\mu}\\
    \vec{\mathcal{F}}\cdot\hat{y} &= \underbrace{\int_{\phi=0}^{2\pi}\sin{\phi}\dd{\phi}}_{0}
    \int_{\mu=-1}^{1} I_\lambda(y, \lambda, t, \mu)\sqrt{(1-\mu^2)}\dd{\mu}\\
    \vec{\mathcal{F}}\cdot\hat{z} &= \underbrace{\int_{\phi=0}^{2\pi}\dd{\phi}}_{2\pi}
    \int_{\mu=-1}^{1} I_\lambda(z, \lambda, t, \mu)\mu\dd{\mu}
  \end{align*}

  So only the $\vec{\mathcal{F}}\cdot\hat{z}$ term may be non-zero.

  We will therefore define this scalar quantity as the \emph{total} [spectral] radiative flux in the limit of plane-parallel atmospheres with symmetry about the radial axis\footnote{Check your geometry and make sure you agree that I can call the radial axis the $z$ axis!}.
  $$
    F_{\lambda, tot} = 2\pi\int_{\mu=-1}^{1} I_\lambda(r, \lambda, t, \mu)\mu\dd{\mu}
  $$

  We can also define the portion of the flux that passes through the surface of a [stellar] atmosphere in the positive radial ($+r$) direction as the surface flux.
  Check where that factor of $2$ goes!

  \begin{equation}
    F_{\lambda, surf} = \pi\int_{\mu=-1}^{1} I_\lambda(r, \lambda, t, \mu)\mu\dd{\mu} \label{eqn:surfFlux}
  \end{equation}

\section{Observational Significance}\label{s:observationSig}
  Imagine a system where an observer measures the energy received by a star a distance, $D$ away.
  Orient the polar/radial/$z$ axis to be along the line-of-sight.
  Also assume that the radius of the star, $r$, is much smaller than $D$ ($r<<D$).
  In this way, all of the observed rays of light from the star can be taken to be parallel.

  Any ray of specific intensity originating from the a surface element ($\dd{^2 S}$) of this star is projected to be parallel to the line-of-sight by a factor, $\mu$. Where $\mu$ is the cosine of the angle between the line-of-sight and the local normal where the specific intensity originated.

  In this way, we completely ignore specific intensities originating perpendicularly from the top and bottom of the star (schematically given in \ref{fig:obsAnnulus}) and from the far-side (left-side in figure) of the star.
  Also, the specific intensity originating from the center of the disk of the star as seen by the observer (the point on the far right-side of the figure) is not affected by the factor $\mu$ whereas those originating from an angle $\theta$ off of this line-of-sight are reduced by $\mu$.

  \begin{figure}[h]
    \centering
    \includegraphics[width=300px]{images/obsSignificance}
    \caption{\label{fig:obsAnnulus} Consider the annulus of differential area $\dd{^2S_{\theta\phi}}$}
  \end{figure}

  The $\dd{^2 S}$ from where the specific intensity originates subtends a differential solid angle element $\dd{^2\Omega} = \frac{\dd{^2 S}}{D^2}$ as seen by the distant observer.

  Therefore, differential spectral radiative flux observed is

  $$
  \dd{F_{\lambda, obs}} = I_\lambda\dd{^2\Omega}
  $$

  On the surface of the star, the differential area element is $\dd{^2 S} = 2\pi r' \dd{r'}$. The radius that locates the annulus, $r'$, is related to the radius of the star by $r' = r\sin{\theta}$, so then $\dd{r'} = r\cos{\theta}\dd{\theta}$.

  So then, the solid angle subtended by this surface area element is ...

  \begin{align*}
    \dd{^2\Omega} &= \frac{2\pi r' \dd{r'}}{D^2}\\
    &= \frac{2\pi r^2 \sin{\theta}\cos{\theta}\dd{\theta}}{D^2}\\
    &= \frac{2\pi r^2 \mu\dd{\mu}}{D^2}
  \end{align*}

  $$
    \dd{^2\Omega} = 2\pi  \frac{r^2}{D^2} \mu\dd{\mu}
  $$

  So the actual spectral flux observed would be equal to ...

  $$
    F_{\lambda, obs} = 2\pi \frac{r^2}{D^2} \int_{0}^{1} I_\lambda(r, \lambda, t, \mu)  \mu\dd{\mu}
  $$

  In the limit of small $\theta$ (equivalent to $r<<D$) $\frac{r}{R} \approx \theta$ and $\frac{r}{R}\approx\frac{\alpha}{2}$.
  Here, $\alpha$ is the angular stellar diameter.

  So the observed spectral flux can be re-written as ...
  $$
    F_{\lambda, obs} = \frac{r^2}{D^2} F_{\lambda, surf} = \frac{\alpha^2}{2^2} F_{\lambda, surf}
  $$

  This enables observers to relate a known distance and radius to and of a distant star to the surface spectral flux of that star. Equivalently, if the star's angular diamter is known.

  Above, the limits on $\mu$ go from $0 \rightarrow 1$ as opposed to the definition in \ref{eqn:surfFlux}. This is accounted for in an assumption that we get $0$ specific intensity from the far-side of the star--the side where $\mu \in [-1,0)$.

  \begin{equation}
    F_{\lambda, obs}(D, \lambda, t) = \frac{\alpha^2}{4} F_{\lambda, surf}(r, \lambda, t)\label{eqn:fluxObs2Surf}
  \end{equation}
