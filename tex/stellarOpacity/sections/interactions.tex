\section{Absorption \& Emission}
  We use the radiation field as a description of the movement of energy through space via electromagnetic radiation. The movement of energy between particles or molecules or bulk matter is handled by the \emph{thermodynamics} of the material.

  These two modes of energy transport can interact with each other.
  Through interactions with matter, energy may be removed from or delivered into the radiation field.
  Absorption interactions remove energy from the radiation field and deliver it to the thermal properties of the material--the \emph{thermal pool}.
  Emission interactions deliver energy to the radiation field derived from this thermal pool.

  Up until now, the radiation field has been agnostic of the particle or wave picture of light.
  We'll now adopt the notion of the photon as the quantization of radiative energy and therefore of the field.
  We know that Planck's Constant ($h$) is given definition by the formula $E_\gamma = \frac{hc}{\lambda}$ where the spectral location or wavelength $\lambda$ is related to the speed of light $c$ by $c=\lambda \nu$, where $\nu$ is another marker of spectral location, called the frequency.
  The energy a photon carries with it may leave the radiation field and therefore diminish the specific intensity of the radiation field through interactions with matter (absorption). Photons may also be generated by the interaction of matter with the radiation field (emission).

  The radiative energy carried by a photon may also be redirected in scattering processes.

  An absorption process is said to thermalize the involved photon--satisfying the interaction between the radiation field and thermal pool. Absorption processes feed energy directly into the thermal kinetic energy of the matter and must therefore couple strongly with thermodynamic properties.

  An emission process generates a photon from the energy of the thermal pool--also satisfying this interaction between the radiation field and thermal pool.

  Absorption and emission processes tend to introduce local equilibrium with respect to energy exchange between the thermal pool (matter) and the radiation field.

  In contrast, scattering processes allow for photons to move through matter without coupling to local thermodynamic properties. Scattering processes therefore tend to delocalize the balance of the matter-radiation equilibrium process and introduce the presence of global properties to any bulk material (i.e., boundaries).

  Detailed understanding of absorption and emission processes require a quantum mechanical interpretation of specific energy landscapes and transitions. This will be the microscopic view of these processes.

  For now, we will focus on the macroscopic analogs of these processes.

  \section{Extinction Coefficient}
    The macroscopic coefficient that describes the removal of energy from the radiation field by matter is named the `extinction coefficient' ($\chi(r, \lambda, t, \mu)$) since radiative energy is `extincted' from the the ray of light.

    It is defined such that an element of material of cross-section $\dd{^2S}$ and length $\dd{s}$ removes from a ray with specific intensity $I_\lambda(r, \lambda, t, \mu)$, incident normal to $\dd{^2S}$ and propogating into solid angle element $\dd{^2\Omega}$, an amount of energy
    \begin{equation}
      \delta E = \chi(r, \lambda, t, \mu)I_\lambda(r, \lambda, t, \mu)\dd{^2S}\dd{s}\dd{\lambda}\dd{t}\dd{^2\Omega}\label{eqn:extinctionC}
    \end{equation}
    in spectral region $\lambda$ of width $\dd{\lambda}$ in a time $\dd{t}$. The extinction coefficient is the product of an atomic absorption cross-section ($m^2$) and the number density of absorbers ($m^{-3}$) summed over all of the states that can interact with photons of wavelength $\lambda$. The dimensions of $\chi$ are therefore $m^{-1}$ with the inverse of $chi$ giving the distance a photon can propogate before it is removed from the ray. This $\frac{1}{\chi}$ is also called the \emph{mean free path} of the photon.

    The extinction properties of a material are generally isotropic. Interaction with moving materials introduces angular dependences due to Doppler Shifts in various frames.

    We say `extinction' instead of `absorption' to include scattering processes. We assume outright that absorption and scattering processes occur independently and add linearly.

    $$
      \chi(r, \lambda, t) = \underbrace{\kappa(r, \lambda, t)}_{Absorption} + \underbrace{\sigma(r, \lambda, t)}_{Scattering}
    $$
